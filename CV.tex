% !TEX TS-program = xelatex
% !TEX encoding = UTF-8 Unicode
% -*- coding: UTF-8; -*-
% vim: set fenc=utf-8

%%%%%%%%%%%%%%%%%%%%%%%%%%%%%%%%%%%%%%%%%%%%%%%%%%%%%%%%%%%%%%%%%
%% SIMPLE-RESUME-CV
%% <https://github.com/zachscrivena/simple-resume-cv>
%% This is free and unencumbered software released into the
%% public domain; see <http://unlicense.org> for details.
%%%%%%%%%%%%%%%%%%%%%%%%%%%%%%%%%%%%%%%%%%%%%%%%%%%%%%%%%%%%%%%%%

% See "README.md" for instructions on compiling this document.

\documentclass[a4paper, MMyyyy,nonstopmode]{simpleresumecv}
% Class options:
% a4paper, letterpaper, nonstopmode, draftmode
% MMMyyyy, ddMMMyyyy, MMMMyyyy, ddMMMMyyyy, yyyyMMdd, yyyyMM, yyyy

%%%%%%%%%%%%%%%%%%%%%%%%%%%%%%%%%%%%%%%%%%%%%%%%%%%%%%%%%%%%%%%%%
%% PREAMBLE.
%%%%%%%%%%%%%%%%%%%%%%%%%%%%%%%%%%%%%%%%%%%%%%%%%%%%%%%%%%%%%%%%%

% CV Info (to be customized).
\newcommand{\CVAuthor}{Павел Захаров}
\newcommand{\CVTitle}{Захаров Павел - резюме}
\newcommand{\CVNote}{Резюме от {\today}}
\newcommand{\CVWebpage}{https://github.com/PaulZakharov}

% PDF settings and properties.
\hypersetup{
pdftitle={\CVTitle},
pdfauthor={\CVAuthor},
pdfsubject={\CVWebpage},
pdfcreator={XeLaTeX},
pdfproducer={},
pdfkeywords={},
unicode=true,
bookmarks=true,
bookmarksopen=true,
pdfstartview=FitH,
pdfpagelayout=OneColumn,
pdfpagemode=UseOutlines,
hidelinks,
breaklinks}

% Shorthand.
\newcommand{\Code}[1]{\mbox{\textbf{#1}}}
\newcommand{\CodeCommand}[1]{\mbox{\textbf{\textbackslash{#1}}}}

%%%%%%%%%%%%%%%%%%%%%%%%%%%%%%%%%%%%%%%%%%%%%%%%%%%%%%%%%%%%%%%%%
%% ACTUAL DOCUMENT.
%%%%%%%%%%%%%%%%%%%%%%%%%%%%%%%%%%%%%%%%%%%%%%%%%%%%%%%%%%%%%%%%%

\begin{document}

%%%%%%%%%%%%%%%
% TITLE BLOCK %
%%%%%%%%%%%%%%%

\Title{\CVAuthor}

\begin{SubTitle}
\href{paltoszaharov@yandex.ru}
{zaharov.ps@yandex.ru}
\,\SubBulletSymbol\,
+7\,999\,960-0317
%\,\SubBulletSymbol\,
\end{SubTitle}

\begin{Body}

%%%%%%%%%%%%%%%
%% EDUCATION %%
%%%%%%%%%%%%%%%

\Section
{Образование}
{Образование}
{PDF:Education}

\Entry
\href{https://mipt.ru/}
{\textbf{Московский физико-технический институт}},
Москва, Россия

\Gap
\BulletItem
Бакалавриат, 
\href{https://mipt.ru/education/chairs/microprocessors/}
{ФРКТ, кафедра микропроцеccорных технологий}
\hfill
\DatestampYMD{2015}{09}{01} --
\DatestampYMD{2019}{07}{01}
\begin{Detail}
\SubBulletItem
Тема диплома: Исследование возможности применения методов
объединения пространственно разнесенных сигналов
в беспроводных сенсорных сетях
%\href{http://www.example.com/my-phd-thesis}
%{A Statistical Approach to Quantifying Climate Change}
\end{Detail}

\BulletItem
Магистратура, 
\href{https://mipt.ru/drec/about/bases/multimedia_technology_and_telecommunications/}
{ФРКТ, кафедра мультимедийных технологий и телекоммуникаций}
\hfill
с \DatestampYMD{2019}{09}{01}

\Gap

\Entry
\href{https://yandexdataschool.ru/}
{\textbf{Школа Анализа Данных, Яндекс}},
Москва, Россия
\Gap
\BulletItem
Инфраструктура больших данных
\hfill
с \DatestampYMD{2019}{09}{01}
%%%%%%%%%%%%%%%%%%%%%%%%%
%% СOURSES %%
%%%%%%%%%%%%%%%%%%%%%%%%%
%\Section
%{Курсы}
%{Курсы}
%{PDF:Курсы}
%
%\BulletItem
%Курсы от Intel:
%\SubBulletItem
%Архитектуры и микроархитектуры универсальных компьютеров
%\SubBulletItem
%Основы программного моделирования ЭВМ
%\SubBulletItem
%Логические элементы и проектирование СБИС
%\SubBulletItem
%Математические основы САПР СБИС
%\Gap
%\BulletItem
%Курсы по машинному обучению:
%\SubBulletItem
%\href{https://vk.com/data_mining_in_action}
%{Курс Data Mining in Action}
%\SubBulletItem
%\href{https://www.coursera.org/learn/supervised-learning/home/welcome}
%{Курс "Обучение на размеченных данных" от МФТИ и Яндекс на базе Coursera}
%%%%%%%%%%%%%%%%%%%%%%%%%
%% RESEARCH EXPERIENCE %%
%%%%%%%%%%%%%%%%%%%%%%%%%
%\Section
%{Research Experience}
%{Research Experience}
%{PDF:ResearchExperience}

%\Entry
%\href{http://www.example.com/my-institute}
%{\textbf{Institute for Advanced Research}},
%Science College

%%%%%%%%%%%%%%%%%%
%% PUBLICATIONS %%
%%%%%%%%%%%%%%%%%%

%\Section
%{Publications}
%{Publications}
%{PDF:Publications}

%\SubSection
%{Journals}
%{Journals}
%{PDF:Journals}

%%%%%%%%%%%%%%%%%%%%%%%%%%%
%% AWARDS & SCHOLARSHIPS %%
%%%%%%%%%%%%%%%%%%%%%%%%%%%

%\Section
%{Awards \&\newline
%Scholarships}
%{Awards \& Scholarships}
%{PDF:AwardsAndScholarships}


%%%%%%%%%%%%%%%%%%%%%%%%%%%%%%%%%%%%%%%%%%%%
%% PROFESSIONAL AFFILIATIONS & ACTIVITIES %%
%%%%%%%%%%%%%%%%%%%%%%%%%%%%%%%%%%%%%%%%%%%%

%\Section
%{Professional Affiliations\newline
%\& Activities}
%{Professional Affiliations \& Activities}
%{PDF:ProfessionalAffiliationsActivities}

%%%%%%%%%%%%%%%%%%%%%%%
%% CAMPUS ACTIVITIES %%
%%%%%%%%%%%%%%%%%%%%%%%

%\Section
%{Campus Activities}
%{Campus Activities}
%{PDF:CampusActivities}

%%%%%%%%%%%%%%%%%%%%%%%%%%%
%% OTHER WORK EXPERIENCE %%
%%%%%%%%%%%%%%%%%%%%%%%%%%%

\Section
{Опыт работы}
{Опыт работы}
{PDF:Работа}

\Entry
\href{https://www.milandr.ru/}
{\textbf{АО "ПКК Миландр"}},
Зеленоград, Москва, Россия

\Gap
\BulletItem
Стажер-исследователь,
отдел разработки ПО,
\hfill
\DatestampYMD{2017}{09}{01} -- 
\DatestampYMD{2019}{08}{30}
\newline
Центр разработки радиоэлектронной аппаратуры
\begin{Detail}
\SubBulletItem
Исследования в области цифровых коммуникационных систем, методов обработки сигналов
\SubBulletItem
Моделирование радиосистем в Simulink
\SubBulletItem
Прототипирование радиосистем с использованием SDR (USRP и LabVIEW)
\SubBulletItem
Обработка экспериментальных данных на MATLAB, Python
\SubBulletItem
Реализация алгоритмов обработки данных на DSP-процессорах
\end{Detail}

%%%%%%%%%%%%%%%
%% LANGUAGES %%
%%%%%%%%%%%%%%%

\Section
{Языки}
{Языки}
{PDF:Языки}

\BulletItem
Английский: Уровень С1.

\Gap
\BulletItem
Немецкий: Уровень A2.

%%%%%%%%%%%%
%% SKILLS %%
%%%%%%%%%%%%

\Section
{Навыки}
{Навыки}
{PDF:Навыки}

\BulletItem
C - Вычислительные программы для ПК, алгоритмы обработки данных на DSP-процессорах;
\Gap
\BulletItem С++ - отдельные задачи (к примеру, \href{https://github.com/PaulZakharov/riscv-in-nutshell}{проект потактового симулятора ЦП архитектуры RISC-V});
\Gap
\BulletItem
MATLAB, LabVIEW - алгоритмы ЦОС;
\Gap
\BulletItem
Python - обработка данных, машинное обучение;
\Gap
\BulletItem
Прочее
\SubBulletItem
Verilog,
\SubBulletItem
{\LaTeX},
\SubBulletItem
MS Office,
\SubBulletItem
Linux

%%%%%%%%%%%%%%%
%% INTERESTS %%
%%%%%%%%%%%%%%%

\Section
{О себе}
{О себе}
{PDF:Осебе}
\BulletItem
Прошел курсы от Intel:
\begin{Detail}
\SubBulletItem
Архитектуры и микроархитектуры универсальных компьютеров
\SubBulletItem
Основы программного моделирования ЭВМ
\SubBulletItem
Логические элементы и проектирование СБИС
\SubBulletItem
Математические основы САПР СБИС
\end{Detail}
\Gap
\BulletItem
Интересы:
Велосипед, 
горные лыжи,
литература.

%%%%%%%%%%%%%%%%
%% REFERENCES %%
%%%%%%%%%%%%%%%%

%\Section
%{References}
%{References}
%{PDF:References}

%\BulletItem
%\textbf{Professor Jonathan Public}
%\newline
%Professor of Geology and Mechanical Engineering
%\newline
%First American University
%\newline
%1000 First Avenue, Springfield, Massachusetts 22222, USA
%\newline
%\href{mailto:jonathanpublic@example.com}
%{jonathanpublic@example.com}
%\,\SubBulletSymbol\,
%+1\,(555)\,222-2222

%%%%%%%%%%%%%%%%%%%%%%%%%%%%%%%%%%%%%%%%
%% THIS IS A SECTION WITH USAGE NOTES %%
%%%%%%%%%%%%%%%%%%%%%%%%%%%%%%%%%%%%%%%%

% Declare a new group to limit the scope of \color to this section.
%\begingroup
%\color{red}
%
%\Section
%{This is a\newline
%Section\newline
%With\newline
%Usage Notes}
%{This is a Section With Usage Notes (For PDF Bookmark)}
%{PDF:ThisIsASectionWithUsageNotes:ForPDFLink}
%
%\SubSection
%{This is a SubSection}
%{This is a SubSection (For PDF Bookmark)}
%{PDF:ThisIsASubSection:ForPDFLink}
%
%\Gap
%\BulletItem
%Use \CodeCommand{Section\{a\}\{b\}\{c\}} and
%\CodeCommand{SubSection\{a\}\{b\}\{c\}}
%to create sections and subsections, where
%\Code{a} is the heading displayed on the page,
%\Code{b} is the PDF bookmark heading, and
%\Code{c} is the internal PDF link (must be unique).
%Sections and subsections will appear in the PDF bookmarks.
%Note the CamelCase command names.
%
%\Gap
%\BulletItem
%Use
%\CodeCommand{Entry},
%\CodeCommand{BulletItem},
%\CodeCommand{SubBulletItem},
%\CodeCommand{Item},
%\CodeCommand{SubItem},
%\CodeCommand{NumberedItem},
%etc.,
%to create entries in the main body of the CV.
%
%\Gap
%\BulletItem
%Enclose entry details between
%\CodeCommand{begin\{Detail\}} and
%\CodeCommand{end\{Detail\}}
%so that they are typeset in a smaller font.
%\begin{Detail}
%\Item
%This is an example of entry detail text enclosed in a \Code{Detail} environment.
%\end{Detail}
%
%\Gap
%\BulletItem
%Use \CodeCommand{Gap} and \CodeCommand{BigGap} to insert vertical spaces between entries to improve layout.
%
%\BigGap
%\SubSection
%{This is Another SubSection}
%{This is Another Subsection (For PDF Bookmark)}
%{PDF:ThisIsAnotherSubSection:ForPDFLink}
%
%\Gap
%\Entry
%This is a plain \CodeCommand{Entry},
%followed by an \CodeCommand{hfill} and a date range
%\hfill
%\DatestampYM{2015}{10} --
%\DatestampYM{2015}{12}
%
%\Gap
%\BulletItem
%This is a \CodeCommand{BulletItem}.
%\Item
%This is an \CodeCommand{Item}, which has no bullet.
%Note the alignment with the \CodeCommand{BulletItem} above.
%
%\Gap
%\SubBulletItem
%This is a \CodeCommand{SubBulletItem}.
%\SubItem
%This is a \CodeCommand{SubItem}, which has no bullet.
%Note the alignment with the \CodeCommand{SubBulletItem} above.
%
%\Gap
%\NumberedItem{[42]}
%This is a \CodeCommand{NumberedItem}.
%Change the value of the macro \CodeCommand{MaxNumberedItem} to adjust the indentation width.
%
%\BigGap
%\SubSection
%{Line, Paragraph, and Page Breaks}
%{Line, Paragraph, and Page Breaks (For PDF Bookmark)}
%{PDF:LineParagraphAndPageBreaks:ForPDFLink}
%
%\Gap
%\BulletItem
%To create a new line within the same paragraph (i.e., preserving the same paragraph indentation), use \CodeCommand{newline} instead of \CodeCommand{\textbackslash};
%the latter will reset the paragraph indentation.
%
%\Gap
%\BulletItem
%To create a new paragraph, use \CodeCommand{par} or simply leave an empty line.
%Paragraph indentations (from
%\CodeCommand{Entry},
%\CodeCommand{BulletItem},
%\CodeCommand{SubBulletItem},
%\CodeCommand{Item},
%\CodeCommand{SubItem},
%\CodeCommand{NumberedItem},
%etc.) do not carry across different paragraphs.
%
%\Gap
%\BulletItem
%To create a new page, use \CodeCommand{newpage}.
%
%\BigGap
%\SubSection
%{Dates}
%{Dates (For PDF Bookmark)}
%{PDF:Dates:ForPDFLink}
%
%\Gap
%\BulletItem
%Use the following macros to specify and display dates consistently:
%\SubBulletItem
%\CodeCommand{DatestampYMD\{yyyy\}\{MM\}\{dd\}}
%(e.g., \CodeCommand{DatestampYMD\{2008\}\{01\}\{15\}})
%\SubBulletItem
%\CodeCommand{DatestampYM\{yyyy\}\{MM\}}
%(e.g., \CodeCommand{DatestampYM\{2008\}\{01\}})
%\SubBulletItem
%\CodeCommand{DatestampY\{yyyy\}}
%(e.g., \CodeCommand{DatestampY\{2008\}})
%
%\Gap
%\BulletItem
%Change the date format option passed to the document class to adjust how dates are displayed throughout the document:
%\SubBulletItem
%\Code{MMMyyyy} (``Jan~2008'')
%\SubBulletItem
%\Code{ddMMMyyyy} (``15~Jan~2008'')
%\SubBulletItem
%\Code{MMMMyyyy} (``January~2008'')
%\SubBulletItem
%\Code{ddMMMMyyyy} (``15~January~2008'')
%\SubBulletItem
%\Code{yyyyMMdd} (``2008-01-15'')
%\SubBulletItem
%\Code{yyyyMM} (``2008-01'')
%\SubBulletItem
%\Code{yyyy} (``2008'')
%
%\endgroup
%
\end{Body}
%
%%%%%%%%%%%%
%% CV NOTE %
%%%%%%%%%%%%
%
\UseNoteFont%
\null\hfill%
[\textit{\CVNote}]
%
\end{document}
